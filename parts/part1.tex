\chapter{哈尔滨理工大学本科毕业设计(论文)撰写规范}
毕业设计(论文)是学生在校学习的最后阶段,是培养学生综合运用所学知识,分析和解决实际工程问题,锻炼创造能力的重要环节。毕业论文是记录科学研究成果的重要文献资料,也是申请学位的基本依据。为了保证我校本科生毕业设计(论文)质量,促进国内外学术交流,特制定《哈尔滨理工大学本科生毕业设计(论文)撰写规范》。
	\section{毕业设计(论文)要求}
我校理工类毕业设计(论文)可分为以下几种类型:工程设计、理论研究、实验研究、计算机软件、综合论文。

对各类毕业设计(论文)具体要求如下:
		\begin{itemize}\denseenum
			\item 工程设计类论文:
				\begin{itemize}\denseenum
					\item 机械类:做此类题目的学生至少要独立完成A0图纸四张(不包括零件图和示意图)和一份设计计算说明书。用计算机进行绘图时,图纸工作量为A0图纸至少2张。
					\item 电  类:学生要独立完成工程(或科研)项目中的全部或相对独立的局部设计、安装、调试工作;要有完整的系统电气原理图或电气控制系统图。
					\item 说明书或论文中一般包括任务的提出,方案论证或文献综述,设计与计算(可分为总体设计和单元设计几部分),要有实验(模拟实验或仿真实验)调试及结果分析,结束语等内容。
				\end{itemize}
			\item 理论研究类论文:工科原则上不提倡理论研究型论文,对该类论文各系要严格把关,选题必须有一定实际意义。正文包括选题的目的、意义,本课题的现状和国内外的研究综述,从而提出问题、分析问题,提出方案、并进行建模、仿真和设计计算等。
			\item 实验研究类论文:学生要独立完成一个完整的实验,取得足够的实验数据。实验要有探索性,而不是简单重复已有的工作。论文应包括文献综述,实验部分讨论与结论等内容。
			\item 计算机软件类论文:学生要独立完成一个应用软件或较大软件中的一个模块,要有足够的工作量,同时要写出软件使用说明书和论文,其格式参照工程设计类论文,当毕业设计(论文)中涉及到有关电路方面内容时,答辩前,必须完成调试试验,要有完整的测试结果和给出各参数指标,并由答辩老师验收;当涉及到有关计算机软件方面的内容时,在答辩前进行计算机演示程序运行情况及给出运行结果,并由答辩老师验收。
			\item 综合类论文:综合类论文要求至少包括前四种类型论文中的三类内容,当有工程设计内容时,图纸工作量酌情减少。
		\end{itemize}

我校文科、管理类专业本科毕业论文可以是理论性论文、应用软件设计或调查报告,其论文形式不能是一些文献资料的简单、机械地堆砌。一篇合格的论文应是一个有内在联系的统一体;论点提出要正确,要有足够的论证依据;论点与论据要协调一致,论据要充分支持论点;要有必要的数据资料及相应的分析、理论、观点。概念表达要准确、清晰;论文要有一定的新意。
	\section{见附录word文档}
	大哈理工毕设要求在附录word文档中,懒得一一写了(:。
