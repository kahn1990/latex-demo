\chapter{画图}
一些简单的画图,代码看parts文件夹里的parts3.tex,看不懂的可以复制粘贴(:
	\section{流程图}
		\subsection{样式一}	
\begin{minipage}[t]{0.8\textwidth}
\scriptsize
 \setlength{\unitlength}{1.8em}
%% start of flow output
\begin{picture}(6.000000,28.000000)(-3.000000,-28.000000)

\put(8.0000,-8.0000){\framebox(6.0000,2.0000)[c]{\shortstack[c]{软件工程师工作日志\\管理平台前端页面}}}
\put(-2.5000,-9.0000){\line(1,0){27.0000}}
%
\put(-4.0000,-12.0000){\framebox(3.0000,2.0000)[c]{\shortstack[c]{用户功\\能模块}}}
\put(-2.5000,-9.0000){\vector(0,-1){1.0000}}
\put(-2.5000,-12.0000){\line(0,-1){1.0000}}
\put(-3.5000,-13.0000){\line(1,0){2.0000}}
\put(-3.5000,-13.0000){\vector(0,-1){1.0000}}
\put(-1.5000,-13.0000){\vector(0,-1){1.0000}}
\put(-4.2500,-20.0000){\framebox(1.5000,6.0000)[c]{\shortstack[c]{用\\户\\注\\册}}}
\put(-2.2500,-20.0000){\framebox(1.5000,6.0000)[c]{\shortstack[c]{用\\户\\登\\录}}}
%
\put(2.5000,-12.0000){\framebox(3.0000,2.0000)[c]{\shortstack[c]{日志周\\报模块}}}
\put(4.0000,-9.0000){\vector(0,-1){1.0000}}
\put(4.0000,-12.0000){\line(0,-1){1.0000}}
\put(3.0000,-13.0000){\line(1,0){2.0000}}
\put(3.0000,-13.0000){\vector(0,-1){1.0000}}
\put(5.0000,-13.0000){\vector(0,-1){1.0000}}
\put(2.2500,-20.0000){\framebox(1.5000,6.0000)[c]{\shortstack[c]{日\\志\\分\\类}}}
\put(4.2500,-20.0000){\framebox(1.5000,6.0000)[c]{\shortstack[c]{日\\志\\详\\情}}}
%
\put(9.5000,-12.0000){\framebox(3.0000,2.0000)[c]{\shortstack[c]{博客文\\章模块}}}
\put(11.0000,-8.0000){\vector(0,-1){2.0000}}
\put(11.0000,-12.0000){\line(0,-1){1.0000}}
\put(10.0000,-13.0000){\line(1,0){2.0000}}
\put(10.0000,-13.0000){\vector(0,-1){1.0000}}
\put(12.0000,-13.0000){\vector(0,-1){1.0000}}
\put(9.2500,-20.0000){\framebox(1.5000,6.0000)[c]{\shortstack[c]{博\\客\\分\\类}}}
\put(11.2500,-20.0000){\framebox(1.5000,6.0000)[c]{\shortstack[c]{博\\客\\详\\情}}}
%
\put(16.5000,-12.0000){\framebox(3.0000,2.0000)[c]{\shortstack[c]{说说状\\态模块}}}
\put(18.0000,-9.0000){\vector(0,-1){1.0000}}
\put(18.0000,-12.0000){\line(0,-1){1.0000}}
\put(17.0000,-13.0000){\line(1,0){2.0000}}
\put(17.0000,-13.0000){\vector(0,-1){1.0000}}
\put(19.0000,-13.0000){\vector(0,-1){1.0000}}
\put(16.2500,-20.0000){\framebox(1.5000,6.0000)[c]{\shortstack[c]{说\\说\\分\\类}}}
\put(18.2500,-20.0000){\framebox(1.5000,6.0000)[c]{\shortstack[c]{说\\说\\详\\情}}}
%
\put(23.0000,-12.0000){\framebox(3.0000,2.0000)[c]{\shortstack[c]{网站导\\航模块}}}
\put(24.5000,-9.0000){\vector(0,-1){1.0000}}
\put(24.5000,-12.0000){\line(0,-1){1.0000}}
\put(23.5000,-13.0000){\line(1,0){2.0000}}
\put(23.5000,-13.0000){\vector(0,-1){1.0000}}
\put(25.5000,-13.0000){\vector(0,-1){1.0000}}
\put(22.7500,-20.0000){\framebox(1.5000,6.0000)[c]{\shortstack[c]{导\\航\\分\\类}}}
\put(24.7500,-20.0000){\framebox(1.5000,6.0000)[c]{\shortstack[c]{导\\航\\详\\情}}}

\end{picture}
%% end of flow output
\end{minipage}
		\subsection{样式二}
\begin{tikzpicture}[node distance=2cm]
 %定义流程图具体形状
\node (start) [startstop] {开始};
\node (in1) [io, below of=start, yshift=-0.5cm] {用户进入首页};
\node (dec1) [decision, below of=in1, yshift=-1.5cm] {是否注册过};
\node (pro1) [process, right of=dec1, xshift=2.6cm] {用户填写注册信息};
\node (dec2) [decision, right of=pro1, xshift=2.5cm] {是否符合规则};
\node (pro2) [process, above of=dec2, yshift=1.5cm] {提示用户填写不合格};
\node (pro3) [process, below of=dec2, yshift=-1.5cm] {注册成功};
\node (out1) [io, below of=dec1, yshift=-1.5cm] {登录网站};
\node (stop) [startstop, below of=out1, yshift=-0.5cm] {结束};
 %连接具体形状
\draw [arrow](start) -- (in1);
\draw [arrow](in1) -- (dec1);
\draw [arrow](dec1) -- node[anchor=east] {no} (pro1);
\draw [arrow](dec1) -- node[anchor=south] {yes} (out1);
\draw [arrow](pro1) -- (dec2);
\draw [arrow](dec2) -- node[anchor=north] {no} (pro2);
\draw [arrow](pro2) -| (pro1);
\draw [arrow](dec2) -- node[anchor=south] {yes} (pro3);
\draw [arrow](pro3) -- (out1);
\draw [arrow](out1) -- (stop);
\end{tikzpicture}
	\subsection{样式三}
		\begin{minipage}[t]{0.8\textwidth}
\begin{tikzpicture}[->,>=stealth',shorten >=1pt,auto,node distance=2.8cm,
                    semithick]
 
  \node[process]         (S1) at (-3, 5)              {$\verb|k_isNote|$};
  \node[state]         (xin1) at (-2, 9)           {$\verb|kt_isNote_id|$};
  \node[state]         (xin2) at (-8, 1)        {$\verb|kt_isNote_date|$};
  \node[state]         (xin3) at (-8, 3)       {$\verb|kt_isNote_content|$};
  \node[state]         (xin4) at (-8, 5)           {$\verb|kt_isNote_title|$};
  \node[state]         (xin5) at (-8, 7)           {$\verb|kt_isNote_year|$};
  \node[state]         (xin6) at (-8, 9)           {$\verb|kt_isNote_month|$};
  \node[state]         (xin7) at (2, -2)           {$\verb|kt_isNote_day|$};
  \node[state]         (xin8) at (2, 9)           {$\verb|kt_isNote_country|$};
  \node[state]         (xin9) at (2, 7)           {$\verb|kt_isNote_author|$};
  \node[state]         (xin10) at (2, 1)           {$\verb|kt_isNote_url|$};
  \node[state]         (xin11) at (2, 3)           {$\verb|kt_isNote_img|$};
  \node[state]         (xin12) at (2, 5)           {$\verb|kt_isNote_isIf|$};
  \node[state]         (xin13) at (-2, 0)           {$\verb|kt_user_id|$};
  \node[state]         (xin14) at (-6, -1)           {$\verb|kt_isTag_id|$};
  \node[process]         (S2) at (-8, -3)              {$\verb|k_isTag|$};
  \node[process]         (S3) at (-3, -5)              {$\verb|k_user|$};

  \path (S1) edge[bend left=26]              node {} (xin1)
            edge[bend left=12]              node {} (xin2)
            edge[bend right=12]             node {} (xin3)
            edge[bend right=26]             node {} (xin4)
            edge[bend left=12]              node {} (xin5)
            edge[bend right=12]             node {} (xin6)
            edge[bend right=26]             node {} (xin7)
            edge[bend left=12]              node {} (xin8)
            edge[bend right=12]             node {} (xin9)
            edge[bend right=26]             node {} (xin10)
            edge[bend left=12]              node {} (xin11)
            edge[bend right=12]             node {} (xin12)
            edge[bend right=26]             node {} (xin13)
            edge[bend right=12]             node {} (xin14)
            (xin13) edge  node {$m/n$} (S3)
        	(xin14) edge  node {$m/n$} (S2);
\end{tikzpicture}
\end{minipage}
\vspace{0.5em}\par
	\centerline{\songti\zihao{-5}{喵喵~}}
	\subsection{样式四}
	\begin{minipage}[t]{0.8\textwidth} 
\tikzstyle{every node}=[anchor=west]
\begin{tikzpicture}[%
  grow via three points={one child at (0.5,-0.7) and
  two children at (0.5,-0.7) and (0.5,-1.4)},
  edge from parent path={(\tikzparentnode.south) |- (\tikzchildnode.west)}]
  \node {ROOT}
    child { node {bin/}}     
    child { node {models/}}
    child { node {node\_modules/}}
	child { node {public/}}
    child { node {routes/}}
    child { node {views/}}          
    child { node {app.js}}
    child { node {db.js}}
    child { node {GruntFile.js}}
    child { node {package.json}}
    child { node {...}};
\end{tikzpicture}	
\end{minipage}
\vspace{0.5em}\par
\centerline{\songti\zihao{-5}{喵~}}
	\section{画表格}
\begin{table}[h]
\renewcommand\arraystretch{1.5}
  \centering
  \caption{日期表}
  \label{tab:tabexamp4}
  \begin{minipage}[t]{0.8\textwidth} 
\begin{tabular}{m{4cm}<{\centering}|m{2cm}<{\centering}|m{2cm}<{\centering}|m{2cm}<{\centering}}
\hline
字段名 & 数据类型 & 长度 & 是否为空  \rule{0pt}{0cm}\\ 
\hline 
\verb|kt_isDate_id| & int & 50 & 否  \rule{0pt}{0cm} \\ 
\hline 
\verb|kt_isDate_date| & varchar & 100 & 否  \rule{0pt}{0cm} \\ 
\hline 
\verb|kt_isNote_id| & int & 50 & 否  \rule{0pt}{0cm} \\ 
\hline 
\verb|kt_user_id| & int & 50 & 否  \rule{0pt}{0cm} \\ 
\hline 
\verb|kt_isNote_isIf| & varchar & 50 & 否  \rule{0pt}{0cm} \\ 
\hline 
\end{tabular}
  \end{minipage}
\end{table}
	\section{代码块}
	Node.js发布脚本代码设计如下所示:
\begin{lstlisting}[language={[ANSI]C},numbers=left,frame=shadowbox,rulesepcolor=\color{gray!2},breaklines=true,backgroundcolor=\color{gray!5},basicstyle=\footnotesize]
#!/bin/sh
#
if [ ! -z "$2" ]; then
   if [ ! -d "$1/$2" ]; then
      mkdir -p "$1/$2"
   else
      echo "[info] shudown app $2"
      cd $1/$2
      forever stop
   
      server_backup_dir=$1/$2.bak
      if [ -d $server_backup_dir ]; then
         rm -rf $server_backup_dir
      fi
      mkdir $server_backup_dir

      echo "[info] backup files in $1/$2"
      tar -czvf $2.bak.tgz $1/$2 > /dev/null 2>&1
      mv $2.bak.tgz $server_backup_dir
      echo "[info] remove old files"
   fi
   cd $1
   tar -xzf $2.tar.gz -C $1/$2/
   echo "[info] start app $2"
   cd $1/$2
   forever start bin/www
fi
\end{lstlisting} 