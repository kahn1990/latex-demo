\documentclass[UTF8,hyperref,oneside]{pkuthss}
%========================产生originauth.tex里的\Square===========================
\usepackage{wasysym}
%=========================提供verbatiminput命令和comment环境===================
\usepackage{verbatim}
\usepackage{array}
\usepackage{tikz,mathpazo}
\usepackage{pgf}
%==============================去掉目录中超级链接的红色颜色改为黑色=================
\usepackage[colorlinks=true,linkcolor=black]{hyperref}
%==============================使用新罗马字体=====================
%\usepackage{times}
%\usepackage{txfonts}
%\usepackage{mathptmx}
%\usepackage[mtbold,mtpluscal,mtplusscr]{mathtime}%数学环境用Times New Roman
%==========================表格设置行间距============================
%\usepackage{setspace}
%===============================代码框=============================
\usepackage{graphicx}  
\usepackage{xcolor}  
\usepackage{listings}  
%========================代码块设定=======网上找的示例,可以有选择的应用============================
%\lstset{%  
%alsolanguage=Java,  
%%language={[ISO]C++},       %language为,还有{[Visual]C++}  
%%alsolanguage=[ANSI]C,      %可以添加很多个alsolanguage,如alsolanguage=matlab,alsolanguage=VHDL等  
%%alsolanguage= tcl,  
%alsolanguage= XML,  
%tabsize=4, %  
%  frame=shadowbox, %把代码用带有阴影的框圈起来  
%  commentstyle=\color{red!50!green!50!blue!50},%浅灰色的注释  
%  rulesepcolor=\color{red!20!green!20!blue!20},%代码块边框为淡青色  
%  keywordstyle=\color{blue!90}\bfseries, %代码关键字的颜色为蓝色,粗体  
%  showstringspaces=false,%不显示代码字符串中间的空格标记  
%  stringstyle=\ttfamily, % 代码字符串的特殊格式  
%  keepspaces=true, %  
%  breakindent=22pt, %  
%  numbers=left,%左侧显示行号 往左靠,还可以为right,或none,即不加行号  
%  stepnumber=1,%若设置为2,则显示行号为1,3,5,即stepnumber为公差,默认stepnumber=1  
%  %numberstyle=\tiny, %行号字体用小号  
%  numberstyle={\color[RGB]{0,192,192}\tiny} ,%设置行号的大小,大小有tiny,scriptsize,footnotesize,small,normalsize,large等  
%  numbersep=8pt,  %设置行号与代码的距离,默认是5pt  
%  basicstyle=\footnotesize, % 这句设置代码的大小  
%  showspaces=false, %  
%  flexiblecolumns=true, %  
%  breaklines=true, %对过长的代码自动换行  
%  breakautoindent=true,%  
%  breakindent=4em, %  
%%  escapebegin=\begin{CJK*}{GBK}{hei},
%%  escapeend=\end{CJK*},  
%  aboveskip=1em, %代码块边框  
%  tabsize=2,  
%  showstringspaces=false, %不显示字符串中的空格  
%  backgroundcolor=\color[RGB]{245,245,244},   %代码背景色  
%  %backgroundcolor=\color[rgb]{0.91,0.91,0.91}    %添加背景色  
%  escapeinside=``,  %在``里显示中文  
%  %% added by http://bbs.ctex.org/viewthread.php?tid=53451  
%  fontadjust,  
%  captionpos=t,  
%  framextopmargin=2pt,framexbottommargin=2pt,abovecaptionskip=-3pt,belowcaptionskip=3pt,  
%  xleftmargin=4em,xrightmargin=4em, % 设定listing左右的空白  
%  texcl=true,  
%  % 设定中文冲突,断行,列模式,数学环境输入,listing数字的样式  
%  extendedchars=false,columns=flexible,mathescape=true  
%  % numbersep=-1em  
%}  
%
%====================================流程图=============================
\usepackage{tikz}
\usetikzlibrary{trees}
\usetikzlibrary{shapes.geometric,arrows, decorations.pathmorphing, backgrounds, positioning, fit, petri, automata}
%==================================设置流程图的节点样式和基本形状===========================
\tikzstyle{startstop} = [rectangle, rounded corners, minimum width=3cm, minimum height=1cm,text centered, draw=black, fill=red!30]
\tikzstyle{io} = [trapezium, trapezium left angle=70, trapezium right angle=110, minimum width=3cm, minimum height=1cm, text centered, draw=black, fill=blue!30]
\tikzstyle{process} = [rectangle, minimum width=3cm, minimum height=1cm, text centered, draw=black, fill=orange!30]
\tikzstyle{decision} = [diamond, minimum width=3cm, minimum height=1cm, text centered, draw=black, fill=green!30]
\tikzstyle{arrow} = [thick,->,>=stealth]
\tikzstyle{every state}=[rectangle, rounded corners, minimum width=1cm, minimum height=1cm,text centered, draw=black, fill=red!30]
%========================================重定义章节标签==============================
%\usepackage[titles]{tocloft}
\usepackage[clearempty,indentafter]{titlesec}
\usepackage{interfaces-titlesec}
%===============================目录章节字体样式=================
\usepackage{titletoc}
\titlecontents{chapter}[0pt]{\color{black}\heiti\zihao{-4}}
              {\contentspush{\thecontentslabel}}
              {}{\titlerule*[8pt]{.}\contentspage\zihao{-4}}
\titlecontents{section}[22pt]{\color{black}\songti\zihao{-4}}
              {\contentspush{\thecontentslabel\ }}
              {}{\titlerule*[8pt]{.}\contentspage\zihao{-4}}
\titlecontents{subsection}[40pt]{\color{black}\songti\zihao{-4}}
              {\contentspush{\thecontentslabel}}
              {}{\titlerule*[8pt]{.}\contentspage\zihao{-4}}
\titlecontents{subsubsection}[45pt]{\color{black}\songti\zihao{-4}}
              {\contentspush{\thecontentslabel}}
              {}{\titlerule*[8pt]{.}\contentspage\zihao{-4}}
%\titlecontents{chapter}[0pt]{\vspace{.5\baselineskip}\bfseries}{第\CJKnumber{\thecontentslabel}章\quad}{}{\hspace{.5em}\titlerule*[10pt]{$\cdot$}\contentspage}
%其中0pt是目录项到版芯左边界的距离。紧跟其后的是在排版目录项之前执行的命令,这里\vspace{.5\baselineskip}表示与上文留出一定的垂直距离,该距离为当前单倍行间距的一半。\bfseries把整条目录项的字体设为黑体。
% ===========================解决目录章节重叠问题=============================
%\makeatletter
%\renewcommand{\numberline}[1]{%
%\settowidth\@tempdimb{#1\hspace{0.5em}}%
%\ifdim\@tempdima<\@tempdimb%
%  \@tempdima=\@tempdimb%
%\fi%
%\hb@xt@\@tempdima{\@cftbsnum #1\@cftasnum\hfil}\@cftasnumb}
%\makeatother
%==============================分别设置中文字体和英文字体=================
\usepackage{xeCJK} 
\usepackage{fontspec,xltxtra,xunicode} 
\setCJKmainfont{宋体} 
\setCJKmonofont{宋体} 
\setmainfont{Times New Roman} 
\definecolor{yellow1}{rgb}{1,0.8,0.2} 
%\usepackage{indentfirst}
%=====================================设定行距========================================
%\linespread{1.3}
\renewcommand{\baselinestretch}{1}
%===================================使引用标记成为上标======================================
\newcommand{\supercite}[1]{\textsuperscript{\cite{#1}}}
%=====================罗列环境中如果每个项目都只有一行左右,则会显得很松散,此时可采用这个命令==========================
\newcommand{\denseenum}{\setlength{\itemsep}{0pt}}
%===========================以上都是基础设置,除样式变化之外不需要改动=======================================
%==============================================================================================================
%=================================================================================================
%=================================================================================================
%=====================================开始设置首页封面等基本文字=======================================
\begin{document}
	%=========================================各种文档信息=================
	\renewcommand{\thesisname}{本科生毕业论文}
	\renewcommand{\thesisnamehead}{学士学位论文}
	%=================================题目一般不宜超过20个字======================
	\title{基于MVC的软件工程师工作日志管理平台的设计与实现}
	\etitle{Software engineers work log management platform based on MVC}
	\author{康健}
	\eauthor{Jian Kang}
	\studentid{1114010609}
	\date{2015年6月16日}
	\school{软件工程学院}
	\major{软件工程}
	\emajor{Software engineering}
	\direction{嵌入式技术}
	\mentor{刘宇鹏}
	\ementor{Yupeng Liu}
	\departmenthead{郭红}
	%=======================================关键词应有3~5个=========================
	\keywords{\heiti 模块化开发\songti 、\heiti MVC\songti 、\heiti Node.js\songti 、\heiti NPM}
	\ekeywords{Modular Development,MVC,Node.js,NPM}
	%===========================以下为正文之前的部分,页码为小写罗马数字,但不显示页眉和页脚========================
	%\frontmatter\pagenumbering{roman}\pagestyle{empty}现在改为大写罗马数字
    \frontmatter\pagenumbering{Roman}
	\maketitle
	%======================================中英文摘要===========================
	% 摘要要求在 3000 字以内。
%{\par\vspace{2em}\par}用来设置空格
\cleardoublepage
\begin{cabstract}{\heiti}

	找了好久,居然发现我大哈理工连latex毕设模板都没有,只好自己动手做一个。
	
	其中,文献一定要引用之后才能显示,使用方法\verb|~\supercite{No1}|,效果如这样~\supercite{No1},完整的编译过程是XeLaTex+BibTex+XeLaTex+XeLaTex。
	
	这里编写中文摘要部分。
	
\end{cabstract}
\cleardoublepage
\begin{eabstract}

	英文摘要部分。
	
\end{eabstract}


	%========================================自动生成目录=========================
	\tableofcontents
	%===============================以下为正文,页码为小写罗马数字,但不显示页眉和页脚====================
	\mainmatter\pagenumbering{arabic}\pagestyle{fancy}
	%=========================================如果写绪言,择在parts文件夹添加======================
	%\include{chap/introduction}
	%===========================================开始各章节===============================================
	\chapter{哈尔滨理工大学本科毕业设计(论文)撰写规范}
毕业设计(论文)是学生在校学习的最后阶段,是培养学生综合运用所学知识,分析和解决实际工程问题,锻炼创造能力的重要环节。毕业论文是记录科学研究成果的重要文献资料,也是申请学位的基本依据。为了保证我校本科生毕业设计(论文)质量,促进国内外学术交流,特制定《哈尔滨理工大学本科生毕业设计(论文)撰写规范》。
	\section{毕业设计(论文)要求}
我校理工类毕业设计(论文)可分为以下几种类型:工程设计、理论研究、实验研究、计算机软件、综合论文。

对各类毕业设计(论文)具体要求如下:
		\begin{itemize}\denseenum
			\item 工程设计类论文:
				\begin{itemize}\denseenum
					\item 机械类:做此类题目的学生至少要独立完成A0图纸四张(不包括零件图和示意图)和一份设计计算说明书。用计算机进行绘图时,图纸工作量为A0图纸至少2张。
					\item 电  类:学生要独立完成工程(或科研)项目中的全部或相对独立的局部设计、安装、调试工作;要有完整的系统电气原理图或电气控制系统图。
					\item 说明书或论文中一般包括任务的提出,方案论证或文献综述,设计与计算(可分为总体设计和单元设计几部分),要有实验(模拟实验或仿真实验)调试及结果分析,结束语等内容。
				\end{itemize}
			\item 理论研究类论文:工科原则上不提倡理论研究型论文,对该类论文各系要严格把关,选题必须有一定实际意义。正文包括选题的目的、意义,本课题的现状和国内外的研究综述,从而提出问题、分析问题,提出方案、并进行建模、仿真和设计计算等。
			\item 实验研究类论文:学生要独立完成一个完整的实验,取得足够的实验数据。实验要有探索性,而不是简单重复已有的工作。论文应包括文献综述,实验部分讨论与结论等内容。
			\item 计算机软件类论文:学生要独立完成一个应用软件或较大软件中的一个模块,要有足够的工作量,同时要写出软件使用说明书和论文,其格式参照工程设计类论文,当毕业设计(论文)中涉及到有关电路方面内容时,答辩前,必须完成调试试验,要有完整的测试结果和给出各参数指标,并由答辩老师验收;当涉及到有关计算机软件方面的内容时,在答辩前进行计算机演示程序运行情况及给出运行结果,并由答辩老师验收。
			\item 综合类论文:综合类论文要求至少包括前四种类型论文中的三类内容,当有工程设计内容时,图纸工作量酌情减少。
		\end{itemize}

我校文科、管理类专业本科毕业论文可以是理论性论文、应用软件设计或调查报告,其论文形式不能是一些文献资料的简单、机械地堆砌。一篇合格的论文应是一个有内在联系的统一体;论点提出要正确,要有足够的论证依据;论点与论据要协调一致,论据要充分支持论点;要有必要的数据资料及相应的分析、理论、观点。概念表达要准确、清晰;论文要有一定的新意。
	\section{见附录word文档}
	大哈理工毕设要求在附录word文档中,懒得一一写了(:。

	\chapter{一级章节设置}
一级章节设置\verb|\chapter|
	\section{二级章节设置}
	二级章节设置\verb|\section|
		\subsection{三级章节设置}
		三级章节设置\verb|\subsection|
		
大致简单效果如上所示,其他一些文本上的样式或者格式,在网上找文档看一下就可以,不一一复制粘贴重复了。
	\chapter{画图}
一些简单的画图,代码看parts文件夹里的parts3.tex,看不懂的可以复制粘贴(:
	\section{流程图}
		\subsection{样式一}	
\begin{minipage}[t]{0.8\textwidth}
\scriptsize
 \setlength{\unitlength}{1.8em}
%% start of flow output
\begin{picture}(6.000000,28.000000)(-3.000000,-28.000000)

\put(8.0000,-8.0000){\framebox(6.0000,2.0000)[c]{\shortstack[c]{软件工程师工作日志\\管理平台前端页面}}}
\put(-2.5000,-9.0000){\line(1,0){27.0000}}
%
\put(-4.0000,-12.0000){\framebox(3.0000,2.0000)[c]{\shortstack[c]{用户功\\能模块}}}
\put(-2.5000,-9.0000){\vector(0,-1){1.0000}}
\put(-2.5000,-12.0000){\line(0,-1){1.0000}}
\put(-3.5000,-13.0000){\line(1,0){2.0000}}
\put(-3.5000,-13.0000){\vector(0,-1){1.0000}}
\put(-1.5000,-13.0000){\vector(0,-1){1.0000}}
\put(-4.2500,-20.0000){\framebox(1.5000,6.0000)[c]{\shortstack[c]{用\\户\\注\\册}}}
\put(-2.2500,-20.0000){\framebox(1.5000,6.0000)[c]{\shortstack[c]{用\\户\\登\\录}}}
%
\put(2.5000,-12.0000){\framebox(3.0000,2.0000)[c]{\shortstack[c]{日志周\\报模块}}}
\put(4.0000,-9.0000){\vector(0,-1){1.0000}}
\put(4.0000,-12.0000){\line(0,-1){1.0000}}
\put(3.0000,-13.0000){\line(1,0){2.0000}}
\put(3.0000,-13.0000){\vector(0,-1){1.0000}}
\put(5.0000,-13.0000){\vector(0,-1){1.0000}}
\put(2.2500,-20.0000){\framebox(1.5000,6.0000)[c]{\shortstack[c]{日\\志\\分\\类}}}
\put(4.2500,-20.0000){\framebox(1.5000,6.0000)[c]{\shortstack[c]{日\\志\\详\\情}}}
%
\put(9.5000,-12.0000){\framebox(3.0000,2.0000)[c]{\shortstack[c]{博客文\\章模块}}}
\put(11.0000,-8.0000){\vector(0,-1){2.0000}}
\put(11.0000,-12.0000){\line(0,-1){1.0000}}
\put(10.0000,-13.0000){\line(1,0){2.0000}}
\put(10.0000,-13.0000){\vector(0,-1){1.0000}}
\put(12.0000,-13.0000){\vector(0,-1){1.0000}}
\put(9.2500,-20.0000){\framebox(1.5000,6.0000)[c]{\shortstack[c]{博\\客\\分\\类}}}
\put(11.2500,-20.0000){\framebox(1.5000,6.0000)[c]{\shortstack[c]{博\\客\\详\\情}}}
%
\put(16.5000,-12.0000){\framebox(3.0000,2.0000)[c]{\shortstack[c]{说说状\\态模块}}}
\put(18.0000,-9.0000){\vector(0,-1){1.0000}}
\put(18.0000,-12.0000){\line(0,-1){1.0000}}
\put(17.0000,-13.0000){\line(1,0){2.0000}}
\put(17.0000,-13.0000){\vector(0,-1){1.0000}}
\put(19.0000,-13.0000){\vector(0,-1){1.0000}}
\put(16.2500,-20.0000){\framebox(1.5000,6.0000)[c]{\shortstack[c]{说\\说\\分\\类}}}
\put(18.2500,-20.0000){\framebox(1.5000,6.0000)[c]{\shortstack[c]{说\\说\\详\\情}}}
%
\put(23.0000,-12.0000){\framebox(3.0000,2.0000)[c]{\shortstack[c]{网站导\\航模块}}}
\put(24.5000,-9.0000){\vector(0,-1){1.0000}}
\put(24.5000,-12.0000){\line(0,-1){1.0000}}
\put(23.5000,-13.0000){\line(1,0){2.0000}}
\put(23.5000,-13.0000){\vector(0,-1){1.0000}}
\put(25.5000,-13.0000){\vector(0,-1){1.0000}}
\put(22.7500,-20.0000){\framebox(1.5000,6.0000)[c]{\shortstack[c]{导\\航\\分\\类}}}
\put(24.7500,-20.0000){\framebox(1.5000,6.0000)[c]{\shortstack[c]{导\\航\\详\\情}}}

\end{picture}
%% end of flow output
\end{minipage}
		\subsection{样式二}
\begin{tikzpicture}[node distance=2cm]
 %定义流程图具体形状
\node (start) [startstop] {开始};
\node (in1) [io, below of=start, yshift=-0.5cm] {用户进入首页};
\node (dec1) [decision, below of=in1, yshift=-1.5cm] {是否注册过};
\node (pro1) [process, right of=dec1, xshift=2.6cm] {用户填写注册信息};
\node (dec2) [decision, right of=pro1, xshift=2.5cm] {是否符合规则};
\node (pro2) [process, above of=dec2, yshift=1.5cm] {提示用户填写不合格};
\node (pro3) [process, below of=dec2, yshift=-1.5cm] {注册成功};
\node (out1) [io, below of=dec1, yshift=-1.5cm] {登录网站};
\node (stop) [startstop, below of=out1, yshift=-0.5cm] {结束};
 %连接具体形状
\draw [arrow](start) -- (in1);
\draw [arrow](in1) -- (dec1);
\draw [arrow](dec1) -- node[anchor=east] {no} (pro1);
\draw [arrow](dec1) -- node[anchor=south] {yes} (out1);
\draw [arrow](pro1) -- (dec2);
\draw [arrow](dec2) -- node[anchor=north] {no} (pro2);
\draw [arrow](pro2) -| (pro1);
\draw [arrow](dec2) -- node[anchor=south] {yes} (pro3);
\draw [arrow](pro3) -- (out1);
\draw [arrow](out1) -- (stop);
\end{tikzpicture}
	\subsection{样式三}
		\begin{minipage}[t]{0.8\textwidth}
\begin{tikzpicture}[->,>=stealth',shorten >=1pt,auto,node distance=2.8cm,
                    semithick]
 
  \node[process]         (S1) at (-3, 5)              {$\verb|k_isNote|$};
  \node[state]         (xin1) at (-2, 9)           {$\verb|kt_isNote_id|$};
  \node[state]         (xin2) at (-8, 1)        {$\verb|kt_isNote_date|$};
  \node[state]         (xin3) at (-8, 3)       {$\verb|kt_isNote_content|$};
  \node[state]         (xin4) at (-8, 5)           {$\verb|kt_isNote_title|$};
  \node[state]         (xin5) at (-8, 7)           {$\verb|kt_isNote_year|$};
  \node[state]         (xin6) at (-8, 9)           {$\verb|kt_isNote_month|$};
  \node[state]         (xin7) at (2, -2)           {$\verb|kt_isNote_day|$};
  \node[state]         (xin8) at (2, 9)           {$\verb|kt_isNote_country|$};
  \node[state]         (xin9) at (2, 7)           {$\verb|kt_isNote_author|$};
  \node[state]         (xin10) at (2, 1)           {$\verb|kt_isNote_url|$};
  \node[state]         (xin11) at (2, 3)           {$\verb|kt_isNote_img|$};
  \node[state]         (xin12) at (2, 5)           {$\verb|kt_isNote_isIf|$};
  \node[state]         (xin13) at (-2, 0)           {$\verb|kt_user_id|$};
  \node[state]         (xin14) at (-6, -1)           {$\verb|kt_isTag_id|$};
  \node[process]         (S2) at (-8, -3)              {$\verb|k_isTag|$};
  \node[process]         (S3) at (-3, -5)              {$\verb|k_user|$};

  \path (S1) edge[bend left=26]              node {} (xin1)
            edge[bend left=12]              node {} (xin2)
            edge[bend right=12]             node {} (xin3)
            edge[bend right=26]             node {} (xin4)
            edge[bend left=12]              node {} (xin5)
            edge[bend right=12]             node {} (xin6)
            edge[bend right=26]             node {} (xin7)
            edge[bend left=12]              node {} (xin8)
            edge[bend right=12]             node {} (xin9)
            edge[bend right=26]             node {} (xin10)
            edge[bend left=12]              node {} (xin11)
            edge[bend right=12]             node {} (xin12)
            edge[bend right=26]             node {} (xin13)
            edge[bend right=12]             node {} (xin14)
            (xin13) edge  node {$m/n$} (S3)
        	(xin14) edge  node {$m/n$} (S2);
\end{tikzpicture}
\end{minipage}
\vspace{0.5em}\par
	\centerline{\songti\zihao{-5}{喵喵~}}
	\subsection{样式四}
	\begin{minipage}[t]{0.8\textwidth} 
\tikzstyle{every node}=[anchor=west]
\begin{tikzpicture}[%
  grow via three points={one child at (0.5,-0.7) and
  two children at (0.5,-0.7) and (0.5,-1.4)},
  edge from parent path={(\tikzparentnode.south) |- (\tikzchildnode.west)}]
  \node {ROOT}
    child { node {bin/}}     
    child { node {models/}}
    child { node {node\_modules/}}
	child { node {public/}}
    child { node {routes/}}
    child { node {views/}}          
    child { node {app.js}}
    child { node {db.js}}
    child { node {GruntFile.js}}
    child { node {package.json}}
    child { node {...}};
\end{tikzpicture}	
\end{minipage}
\vspace{0.5em}\par
\centerline{\songti\zihao{-5}{喵~}}
	\section{画表格}
\begin{table}[h]
\renewcommand\arraystretch{1.5}
  \centering
  \caption{日期表}
  \label{tab:tabexamp4}
  \begin{minipage}[t]{0.8\textwidth} 
\begin{tabular}{m{4cm}<{\centering}|m{2cm}<{\centering}|m{2cm}<{\centering}|m{2cm}<{\centering}}
\hline
字段名 & 数据类型 & 长度 & 是否为空  \rule{0pt}{0cm}\\ 
\hline 
\verb|kt_isDate_id| & int & 50 & 否  \rule{0pt}{0cm} \\ 
\hline 
\verb|kt_isDate_date| & varchar & 100 & 否  \rule{0pt}{0cm} \\ 
\hline 
\verb|kt_isNote_id| & int & 50 & 否  \rule{0pt}{0cm} \\ 
\hline 
\verb|kt_user_id| & int & 50 & 否  \rule{0pt}{0cm} \\ 
\hline 
\verb|kt_isNote_isIf| & varchar & 50 & 否  \rule{0pt}{0cm} \\ 
\hline 
\end{tabular}
  \end{minipage}
\end{table}
	\section{代码块}
	Node.js发布脚本代码设计如下所示:
\begin{lstlisting}[language={[ANSI]C},numbers=left,frame=shadowbox,rulesepcolor=\color{gray!2},breaklines=true,backgroundcolor=\color{gray!5},basicstyle=\footnotesize]
#!/bin/sh
#
if [ ! -z "$2" ]; then
   if [ ! -d "$1/$2" ]; then
      mkdir -p "$1/$2"
   else
      echo "[info] shudown app $2"
      cd $1/$2
      forever stop
   
      server_backup_dir=$1/$2.bak
      if [ -d $server_backup_dir ]; then
         rm -rf $server_backup_dir
      fi
      mkdir $server_backup_dir

      echo "[info] backup files in $1/$2"
      tar -czvf $2.bak.tgz $1/$2 > /dev/null 2>&1
      mv $2.bak.tgz $server_backup_dir
      echo "[info] remove old files"
   fi
   cd $1
   tar -xzf $2.tar.gz -C $1/$2/
   echo "[info] start app $2"
   cd $1/$2
   forever start bin/www
fi
\end{lstlisting} 
	%=============================================结论=============================
	%===========================以下为正文之后的部分,页码为大写罗马数字=========================
	\backmatter\pagenumbering{Roman}
	\chapter{结论}
这里进行结论,欢迎后续有同学进一步完善。



	% ============================================参考文献=========================================
	\begin{appendix}{\songti\zihao{-4}}
		%==================================设置参考文献========================================
		\bibliographystyle{chinesebst}
		\bibliography{bookBib}
		%==================================此命令手动地在目录中增加相当于章级别的一行=================
		\addcontentsline{toc}{chapter}{参考文献}{\heiti\zihao{-2}\centering}
		\phantomsection
	\end{appendix}
	%=========================================致谢==========================================
	\chapter{致谢}
这里进行鸣谢


	%===========================================附录====================================================
	%\cleardoublepage
\chapter{附录A}
\leftline{\zihao{-4}{英文原文}}
%\centerline{\songti\zihao{-5}{图5-3 生活状态流程图}}
\section*{\centerline{英文附录标题}}
\linespread{1.2}
在这里编写英文附录
\vfill
\clearpage
\leftline{\zihao{-4}{中文译文}}
\section*{\centerline{中文附录标题}}
\linespread{1}
	在这里编写中文附录

\end{document}

